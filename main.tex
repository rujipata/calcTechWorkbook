
\documentclass[11pt,a4paper,oneside]{book}
\usepackage[utf8]{inputenc}
\usepackage[english]{babel}
\usepackage[a4paper, left=28mm,right=28mm,top=20mm]{geometry}
\usepackage{setspace}
\usepackage[fleqn]{amsmath,bm}%[fleqn] for using flalign* environment (make equation flush left);see 'differentiability' lesson

\usepackage[normalem]{ulem} %strikeout text using black strikeout line. The option [normalem] prevents \emph to turn to underline. The command is \sout{}
\newcommand{\msout}[1]{\text{\sout{\ensuremath{#1}}}} % strikeout math mode uses the command,for example, $\msout{\mathsf{stuckout}}$. Requiring amsmath package and ulem package.
\newcommand\redout{\bgroup\markoverwith {\textcolor{red}{\rule[.5ex]{2pt}{0.8pt}}}\ULon}%change the color and the thickness of a strikethrough. Use the command \redout{} instead of \sout{} to use the strikeout line in red. Note:\rule[raise]{width}{thickness}, so change the number in the third argument in \rule.

\usepackage{graphicx}
\usepackage{tikz}
\usepackage{pgfplots, pgfplotstable}
%\usepackage{subfig}
\usepackage{subcaption}
\usepackage{wrapfig}
%\usepackage{float}
\graphicspath{ {./images/} }
\usepackage[dvipsnames]{xcolor} %see details from https://www.overleaf.com/learn/latex/Using_colours_in_LaTeX
\usepackage{hyperref}%clickable url
\hypersetup{
    colorlinks=true,
    linkcolor=blue,
    citecolor=blue,
    filecolor=blue,      
    urlcolor=blue,
}
\usepackage{xurl}%break url in multiple lines


 
\usepackage[inline]{enumitem} 
\usepackage{kantlipsum}%set no indent when using enumerate

\newenvironment{enumInline1}{
\renewcommand{\labelenumi}{\textbf{(\alph{enumi})}}
\begin{enumerate*}[itemjoin={{\quad}}]
}
  %\setlength{\itemsep}{1pt}
  %\setlength{\parskip}{1pt}
  %\setlength{\parsep}{1pt}
{\end{enumerate*}}

\usepackage{import}
\usepackage{multicol}
%\setlength{\columnsep}{1}
\usepackage{vwcol}
\usepackage{tasks}
\usepackage{comment}
\usepackage{answers}
\Newassociation{sol}{Solution}{ans}
\Newassociation{solL}{Solution}{ansL}
\usepackage{amsthm}%required to have normal font (rather than italic) in examples
\theoremstyle{definition}%required to have normal font (rather than italic) in examples
\newtheorem{example}{Example}[chapter]
\begin{comment}
\newcounter{example}[section]
\newenvironment{example}[1][]{\refstepcounter{example}\par\medskip\noindent \textbf{Example~\theexample #1} \rmfamily}{\medskip}
\end{comment}
%Tables: 
\usepackage{array}
\setlength{\arrayrulewidth}{0.5mm}%set the thickness of the borders 
%\setlength{\tabcolsep}{3pt} %set space between text and right/left border
%\renewcommand{\arraystretch}{1.5}
\usepackage{subcaption} % required for 'subtable' environment and 'subfigure' environment
\usepackage{tabularx}
\usepackage{caption}
\usepackage{ragged2e}%Make each line of the caption will be moved to the left margin, too.  Butthis time the command \RaggedRight
\captionsetup{
  font=footnotesize,
}


\usepackage{floatrow}
\usepackage{makecell}
\usepackage{booktabs}

%Figure
\usepackage[export]{adjustbox} %change the default alignment of a image from left or right

%change the word 'Chapter'
\usepackage{fancyhdr}
\renewcommand{\chaptername}{Lesson}

%customize chapter

\usepackage{titlesec}

\titlespacing{\chapter}{*-3}{0pt}{0pt}
\titleformat
{\chapter} % command
[display] % shape
{\bfseries\Large\itshape} % format
{Lesson No. \ \thechapter} % label {Lesson No. \ \thechapter}
{-0.5ex} % sep: horizontal separation between label and title body
{
    \rule{\textwidth}{1pt}
    \vspace{1.5ex}
    \centering
} % before-code: code preceding the title body

%set headers (requiring [pagestyles] with 'titlesec' package OR usepackage{titleps}
\newpagestyle{main}{
  \sethead[\thepage][\chaptertitle][(\thesection] % even
          {\emph{LESSON NO.} \emph{\thechapter}}{\sectiontitle}{\thepage}} % odd
\pagestyle{main}
%To set footers, use \setfoot instead of \sethead

\begin{comment}
[
\vspace{-0.5ex}%
\rule{\textwidth}{0.3pt}
] % after-code
\end{comment}

%Ruler
\newcommand{\HRule}{\rule{\linewidth}{0.5mm}}

%Box
\usepackage{fancybox,framed}
\usepackage[many]{tcolorbox}


%Biography
\usepackage[
backend=biber,
style=numeric,
citestyle=authortitle
]{biblatex}

\usepackage{cancel}
\newcommand\Ccancel[2][black]{\renewcommand\CancelColor{\color{#1}}\cancel{#2}}

\addbibresource{math1100workbook.bib} %Imports bibliography file

\begin{document}

\frontmatter
\import{./}{title.tex}

\clearpage
\thispagestyle{empty}

\tableofcontents
%\listoffigures
%\listoftables
%\newpage
%\vspace{.25in}

\mainmatter
%%%%%%%%%%%%%%First Part%%%%%%%%%%%%%%%%%%%%%%%%%%
\part{\emph{Derivatives}}
\newpage

\chapter{Limits }\label{limits}
\import{derivatives/}{limits.tex}
\vspace{-0.3in}
%\section*{{Exercises \ref{limits}}}
%\import{Exercises/}{exercises_limits.tex}

\chapter{Derivatives of Functions}\label{introDerv}
\import{derivatives/}{derivatives-functions.tex}

\chapter{Differentiability and Continuity}\label{differentiability}
\import{derivatives/}{Differentiability.tex}

\chapter{Basic Rules of Differentiation and Marginal Analysis}\label{diffRules}
\import{derivatives/}{differentiationRules.tex}

\chapter{Chain Rule and General Power Rule}\label{GenPower}
\import{derivatives/}{GeneralPowerRule.tex}

\chapter{Related Rates}\label{RelatedRates}
\import{derivatives/}{RelatedRates.tex}

\chapter{First Derivative Test For Local Extreme Values}\label{FirstDerv}
\import{derivatives/}{FirstDerivativeTest.tex}

\chapter{Global Extreme Values}\label{GlobalExtreme}
\import{derivatives/}{GlobalExtremeValues.tex}

\chapter{Second Derivative and Concavity}\label{concavity}
\import{derivatives/}{2ndDervConcavity.tex}

%\chapter{Derivatives and the Shape of a Graph}\label{shape}
%\import{derivatives/}{shapeOfGraph.tex}

%\chapter{Optimization}\label{optimization}
%\import{derivatives/}{optimization.tex}

\chapter{Applied Optimization}\label{appliedOptimization}
\import{derivatives/}{appliedOptimization.tex}

%\chapter{Curve Sketching}\label{curveSketch}
%\import{derivatives/}{curveSketch.tex}

\chapter{Product and Quotient Rules}\label{productQuotient}
\import{derivatives/}{productQuotient.tex}

\chapter{Implicit Differentiation}\label{implicitDiff}
\import{derivatives/}{implicitDiff.tex}

%%%%%%%%%%%%%%Second Part%%%%%%%%%%%%%%%%%%%%%%%%%%
\part{\emph{Derivatives of Exponential and Logarithmic Functions}}

\chapter{Derivatives of Natural Exponential Functions}\label{expoFun}
\import{ExpoLog/}{expoFun.tex}

\chapter{Derivatives of Natural Logarithm Functions}\label{DervNLog}
\import{ExpoLog/}{DervNLog.tex}

\chapter{Natural Logarithm Functions: Properties and Differentiation}\label{NLogPropDerv}
\import{ExpoLog/}{NLogPropDerv.tex}

\chapter{Exponential Growth and Decay Models}\label{growthDecay}
\import{ExpoLog/}{growthDecay.tex}

\chapter{Continuously Compounded Interest}\label{interest}
\import{ExpoLog/}{interest.tex}

\chapter{Applications of Natural Log Functions}\label{appNlog}
\import{ExpoLog/}{appNlog.tex}

\chapter{Functions of Two Variables and Partial Derivatives}\label{PartialDerivatives}
\import{Two Variables/}{PartialDerivatives2.tex}



%\chapter{Exponential Models}\label{expModel}
%\import{ExpoLog/}{expModel.tex}

%%%%%%%%%%%%%%Third Part%%%%%%%%%%%%%%%%%%%%%%%%%%
\part{\emph{Integration}}

\chapter{Antidifferentiation}\label{antiderv}
\import{Integration/}{antiderv.tex}

\chapter{Definite Integrals and Net Change of a Function}\label{defIntg}
\import{Integration/}{defIntg.tex}

\chapter{Definite Integrals and Area}\label{defIntgArea}
\import{Integration/}{defIntgArea1.tex}

\chapter{Integration by Substitution}\label{IntgSub}
\import{Integration/}{IntgSub.tex}

\chapter{Future Value and Present Value}\label{presentValue}
\import{Integration/}{presentValue.tex}

%\chapter{Average Value of A Function}\label{defIntgApp}
%\import{Integration/}{defIntgApp.tex}

%\chapter{Improper Integrals}\label{improper}
%\import{Integration/}{improperIntg.tex}

%\chapter{Derivatives as Rates of Change}\label{rateChange}
%\import{derivatives/}{rate-of-change.tex}

%\chapter{Natural Logarithm Functions}\label{naturalLog}
%\import{ExpoLog/}{naturalLog.tex}

%\chapter{Improper Integrals}\label{improperIntg}
%\import{Integration/}{improperIntg.tex}

%\part{Example Instructions and Study Help}
%\chapter{Exam 2 Study Help and Instructions}\label{Exam2studyHelp} 
%\import{Exam Instructions/}{Exam2StudyHelpInstructions.tex}


\part{Example Solutions}
\import{solutions/}{Example_Solutions.tex}

%\part*{Additional Resources for Homework Assignments}
%\import{Resources/}{HomeworkResources1.tex}
%\newpage
%\import{Resources/}{HomeworkResources2.tex}
%\newpage
%\import{Resources/}{HomeworkResources3.tex}

%\part*{Teaching Resources}
%\import{Resources/}{teaching_resources.tex}

\printbibliography
\end{document}

