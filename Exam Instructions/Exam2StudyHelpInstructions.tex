%Dave's Exam I Topics and Exam II Topics

{
\noindent
    \rule{\textwidth}{1pt}
    \vspace{1ex}
    %\centering
}
\vspace{-0.35 in}
\subsection*{Exam Coverage:} 
{
\noindent
    \rule{\textwidth}{1pt}
    
    %\centering
}
\noindent This exam will covers the materials from \textbf{lessons 8-12}. See topical coverage below for more details.\\
%%%%%%%%%%%%%%%%%%%%%%%%%%%%%%%%%%%%%%%%%%%%%%%%%%%%%%%%
{
\noindent
    \rule{\textwidth}{1pt}
    \vspace{0ex}
    %\centering
}
  \vspace{-0.25in}
\subsection*{Study Help (Suggestion):} 
{
\noindent
    \rule{\textwidth}{1pt}
    \vspace{1ex}
    %\centering
}
\noindent Look over the old exams (below), lecture handouts, homework assignments, quizzes.
\begin{itemize}[leftmargin=*]
\item \emph{Old Exam 1}: Fall 2017 (ignore questions 1-3 and 8), Spring 2018 (ignore questions 1-3 and 8) and Fall 2018 ((ignore questions 1-3 and 7)).
\item \emph{Old Exam 2}: Fall 2017 , Spring 2018, Fall 2018 and Spring 2019.
\end{itemize}
%%%%%%%%%%%%%%%%%%%%%%%%%%%%%%%%%%%%%%%%%%%%%%%%%%%%%%%%%%%%
{
\noindent
    \rule{\textwidth}{1pt}
    \vspace{1ex}
    %\centering
}
  \vspace{-0.25in}
\subsection*{Topical Coverage:}
{
\noindent
    \rule{\textwidth}{1pt}
    \vspace{1ex}
    %\centering
}
\begin{itemize}[leftmargin=*]
\item Derivatives; horizontal tangent lines; relative extreme points.
\item Derivatives; increasing/decreasing intervals for a function; analyzing the end behavior of the derivative.
\item Demand equation; cost, revenue, and profit functions; absolute maximum value.
\item Derivatives, marginal cost, rates of change and units.
\item $1^{st}$ and $2^{nd}$ derivatives; rates of change and units; absolute minimum value.
\item Combining Differentiation Rules, increasing/decreasing intervals, relative extreme points, sign chart.
\item Time rate of change (application using the chain rule: related rates)
\item Related Rates and implicit differentiation.
\item Application:  Quotient Rule. 
\item Economic application: revenue function, marginal revenue function, maximum revenue.
\end{itemize}
%%%%%%%%%%%%%%%%%%%%%%%%%%%%%%%%%%%%%%%%%%%%%%%%%%%%%%%%%%%%%%
\newpage
{
\noindent
    \rule{\textwidth}{1pt}
    \vspace{1ex}
    %\centering
}
  \vspace{-0.25in}
\subsection*{Instructions}
{
\noindent
    \rule{\textwidth}{1pt}
    \vspace{1ex}
    %\centering
}
\begin{itemize}[leftmargin=*]
    \item You may \textit{not} use your books, or notes on this exam.
    \item \underline{ONLY} a physical basic-calculator and writing utensils are allowed.
    \item Carefully follow the instruction given on each question.
    \item \textbf{IF instructed}, show your work in the space on each question by following the rules below:
\begin{itemize}[leftmargin=*]
    \item \textbf{Organize your work}, in a reasonably neat and coherent way, in the space provided. Work scattered all over the page without a clear ordering will receive very little credit. 
    \item \textbf{Mysterious or unsupported answers will not receive full credit}.  A correct answer, unsupported by calculations, explanation,or algebraic work will receive no credit; an incorrect answer supported by substantially correct calculations and explanations might still receive partial credit.
    \item \textbf{Identify your final answer:} Circle or underline your final answer for each question. %Write down your final answer in the blank (if shown) at the bottom right on the space in each question.
    \item \textbf{More space:} If you need more space, ask your proctor for an additional paper; clearly indicate when you have done this. Write down your name and A-number on the paper.
\end{itemize}
\item Some formulas and/or some rules will be provided during the test. See the formulas below.
\end{itemize}
%%%%%%%%%%%%%%%%%%%%%%%%%%%%%%%%%%%%%%%%%%%%%%%%%%%%%%%%%
{
\noindent
    \rule{\textwidth}{1pt}
    \vspace{1ex}
    %\centering
}
 \vspace{-0.25in}
\subsection*{Formulas and Rules:}
{
\noindent
    \rule{\textwidth}{1pt}
    \vspace{1ex}
    %\centering
}

The following formulas will be given during the exam:
\begin{itemize}[leftmargin=*]
\item \textbf{Basic Differentiation Rules:} In what follows, $f(x)$ and $g(x)$ are differentiable functions of $x$ and $k$ is a constant.
\begin{itemize}
    \item Constant Multiple Rule: $\displaystyle\frac{d}{dx}(kf(x))=kf'(x)$
    \item Power Rule: $\displaystyle\frac{d}{dx}(x^n)=nx^{n-1}$
    \item Constant Rule: $\displaystyle\frac{d}{dx}(k)=0$
    \item Sum Rule: $\displaystyle\frac{d}{dx}(f(x)+g(x))=f'(x)+g'(x)$
     \item Difference Rule: $\displaystyle\frac{d}{dx}(f(x)-g(x))=f'(x)-g'(x)$\\
\end{itemize}
\item \textbf{Chain Rule:} If $h(x)=(f\circ g)(x)$, $h'(x)=f'(g(x))\cdot g'(x)$\\
\item \textbf{General Power Rule:} If $h(x)=[g(x)]^n$, $h'(x)=n[g(x)]^{n-1}\cdot g'(x)$\\
\item \textbf{Product Rule:} If $h(x)=f(x)g(x)$, then $h'(x)=f'(x)g(x)+g'(x)f(x)$.\\
\item \textbf{Quotient Rule:} If $h(x)=\displaystyle\frac{f(x)}{g(x)}$, then $h'(x)=\displaystyle\frac{f'(x)g(x)-g'(x)f(x)}{[g(x)]^2}$.\\
\item \textbf{Quadratic Formula:} Given a quadratic function $f(x)=ax^2+bx+c$, $x=\displaystyle\frac{-b\pm \sqrt{b^2-4ac}}{2a}$
\item \textbf{Revenue Function:} $R(x)=x\cdot p$, where $p$ represents the demand equation.
\item \textbf{Revenue Function:} $P(x)=R(x)-C(x)$
\end{itemize}


