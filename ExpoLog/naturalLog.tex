%Dave's handout 1.7 intro 2nd derivatives
%Dave's handout 2.2 2nd derivative rule
\vspace{-0.25 in}
\begin{framed}
\subsection*{Objectives}
\begin{itemize}
    \item Understand how and when to use the \textbf{inverse properties} for natural exponential and natural log functions.
    \item Be able to \textbf{simplify expressions} involving natural exponential and natural log functions.
    \item Be able to \textbf{solve equations} involving natural exponential and natural log functions.
\end{itemize}

%%%Reading Assignment%%%
\subsection*{Suggested Reading:}
\begin{itemize}
\item \cite{Calaway}\footnotemark[1]
   \begin{itemize}
        \item \emph{Section 1.8 Logarithmic Functions}
    \end{itemize}

\item \cite{openstaxColAlgebra}\footnotemark[2]
    \begin{itemize}
        \item \emph{Section 6.3 Logarithmic Functions}
       \begin{itemize}
           \item Using Natural Logarithms
       \end{itemize}
       \item \emph{Section 6.6 Exponential and Logarithmic Equations}
       \begin{itemize}
           \item Solving Exponential Equations Using Logarithms
       \end{itemize}
    \end{itemize}

\end{itemize}
%\subsection*{Supplemental Materials:}
%%%Key Terms%%%
\subsection*{Key Terms and Concepts:} 

\begin{multicols}{2}
\begin{itemize}
    \item Natural log functions (base $e$)
    \item Inverse Properties
    \item Solving exponential and log equations.
\end{itemize}
\end{multicols}
\end{framed}
\footnotetext[1]{Available free to download from \url{http://www.opentextbookstore.com/details.php?id=14} .}
\footnotetext[2]{Available free to download from \url{https://openstax.org/details/books/college-algebra} .}
\newpage
%%%%%%%%%%START LESSON CONTENT%%%%%%%%%%%%%
%\noindent\makebox[\linewidth]{\rule{\textwidth}{0.8pt}}
\Opensolutionfile{ans}[ans15]
\Opensolutionfile{ansL}[ansL15]
%%%%%%%%%%%%%%%%Start First Topic%%%%%%%%%%%%%%%%%%%%%%%%%%%%%
\noindent The most frequently used base for logarithms is $e$. Base $e$ logarithms are important in calculus and some scientific applications; they are called \textbf{natural logarithms}. 

\noindent This lesson provides a brief review of the natural $log$ function $f(x)=ln(x)$ and  its two important inverse properties. We use these properties often when solving equations involving exponential and $log$ terms.
\begin{tcolorbox}[title = {Definition of The Natural Logarithms}]
The logarithm with \textbf{base \textit{e}} is called the \textbf{\textit{natural} logarithm} and is denoted by $\ln (x)$.\\ That is,

$$\ln(x) = \log_e (x)$$ 
Thus, 
$$ y= \ln(x)\quad \text{if and only if}\quad  x = e^y$$

$e \approx 2.718281828459045...$
\end{tcolorbox}
%%%%%%%%%%%%%%Dave's handout 4.4. Natural Logarithm Functions%%%%%%%%%%%
\noindent Recall that the natural log function $f(x)=ln(x)$  is the inverse of the exponential function $g(x)=e^x$.  As such, we have two important inverse properties:    

\begin{tcolorbox}[title = {Inverse Properties of Logarithms}]

Since the functions $y=e^x$ and $y=ln(x)$ are inverse functions,
\begin{equation}\label{eq:inverseLog1}
    ln(e^x)=x \quad \text{for all}\quad x
\end{equation}
and
\begin{equation}\label{eq:inverseLog2}
    e^{ln(x)}=x \quad \text{for}\quad x>0
\end{equation}

\end{tcolorbox}
%%%%%%%%%%%%%%%%%%%%%%%%%%%%%%%%%%%%%%%%%%%%%
%%%%%%%%%%%%%%%%%Openstax College Algebra 6.3 Logarithmic Functions%%%%%%%%%%%%%%%%%%%%%%%%%%
\noindent Most values of $ln(x)$ can be found only using a calculator. The major exception is that, because the logarithm of 1 is always 0 in any base, $ln(1)=0$. For other natural logarithms, we can use the $ln$ key that can be found on most scientific calculators.  We can also find the natural logarithm of any power of $e$ using the inverse property of logarithms in equation \ref{eq:inverseLog1}. 
In addition, we use these properties often when solving equations involving exponential and log terms.\\%Dave's handout 4.4. Natural Logarithm Functions%
%%%%%%%%%%%%Dave's handout 4.4. Natural Logarithm Functions%%%%%%%%%
%%%Examples%%%
\begin{example}
Simplify the expression, making use of laws of exponents: $e^{4ln(x)}$
    %%short answer
    \begin{sol}
    $x^4$
    \end{sol}
    %%solution
    \begin{solL}
    Complete solution here.....
    
    \end{solL}
    
\end{example}
\vspace*{\stretch{1}}
%%%%%%%%%%%%%%%%%%%%%%%%
\begin{example}
Simplify the expression, making use of laws of exponents: $e^{ln(x)-ln(3)}$
    %%short answer
    \begin{sol}
    $\displaystyle\frac{x}{3}$.
    \end{sol}
    %%solution
    \begin{solL}
    Complete solution here.....
    
    \end{solL}
    
\end{example}
\vspace*{\stretch{1}}
%%%%%%%%%%%%%%%%%%%%%%%%
\begin{example}
Solve for $x$: $ln(x^2)=10$
    %%short answer
    \begin{sol}
    $x=\pm \sqrt{e^{10}}$ or $x=\pm e^5$.
    \end{sol}
    %%solution
    \begin{solL}
    Complete solution here.....
    
    \end{solL}
    
\end{example}
\vspace*{\stretch{1}}
\newpage
%%%%%%%%%%%%%%%%%%%%%%%%
\begin{example}
Solve for $x$: $4ln(3x)=9$
    %%short answer
    \begin{sol}
    $x=\displaystyle\frac{1}{3}e^{9/4}$.
    \end{sol}
    %%solution
    \begin{solL}
    Complete solution here.....
    
    \end{solL}
    
\end{example}
\vspace*{\stretch{1}}
%%%%%%%%%%%%%%%%%%%%%%%%
\begin{example}
Solve for $x$: $ln\left(\displaystyle\frac{3}{x}\right)=2$
    %%short answer
    \begin{sol}
    $x=\displaystyle\frac{3}{e^2}$.
    \end{sol}
    %%solution
    \begin{solL}
    Complete solution here.....
    
    \end{solL}
    
\end{example}
\vspace*{\stretch{1}}
%%%%%%%%%%%%%%%%%%%%%%%%
\begin{example}
Solve for $x$: $2ln(x+1)=5$
    %%short answer
    \begin{sol}
    $x=e^{5/2}-1$.
    \end{sol}
    %%solution
    \begin{solL}
    Complete solution here.....
    
    \end{solL}
    
\end{example}
\vspace*{\stretch{1}}
\newpage
%%%%%%%%%%%%%%%%%%%%%%%%
\begin{example}
Given $f(x)=2x-4e^x$, find the coordinates of any relative maximum or relative minimum points. 
    %%short answer
    \begin{sol}
    $(ln(0.5),2ln(0.5)-2)$.
    \end{sol}
    %%solution
    \begin{solL}
    Complete solution here.....
    
    \end{solL}
    
\end{example}
\vspace*{\stretch{2}}
%%%%%%%%%%%%%%%%%%%%%%%%
\begin{example}
Find the point on the graph of the function $f(x)=e^{2x}$ where the slope of graph is equal to 4.
    %%short answer
    \begin{sol}
    $\left(\displaystyle\frac{ln(2)}{2},2\right)$.
    \end{sol}
    %%solution
    \begin{solL}
    Complete solution here.....
    
    \end{solL}
    
\end{example}
\vspace*{\stretch{1}}
%%%%%%%%%%%%%%%End Lesson%%%%%%%%%%%%%%%%%%
\Closesolutionfile{ans}
\Closesolutionfile{ansL}

%%%Short Answers to Examples%%%
%\newpage
\vspace*{\fill}

\subsection*{Short Answers to Examples}
%\vspace{-0.25cm}
\begin{multicols}{2}
\input{ans15}
\end{multicols}


