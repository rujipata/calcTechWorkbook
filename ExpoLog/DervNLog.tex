%Dave's handout 4.5 Derivatives of Natural Logarithm Functions
\vspace{-0.25 in}
\begin{framed}
\subsection*{Objectives}
\begin{itemize}
    \item Understand the techniques for differentiating log functions.
    \item Be able to analyze functions involving log functions.
\end{itemize}

%%%Reading Assignment%%%
\subsection*{Suggested Reading:}
\begin{itemize}
\item \cite{Calaway}\footnotemark[1]
   \begin{itemize}
        \item \emph{Section 2.4 Product and Quotient Rules}
        \begin{itemize}
            \item Example 3
        \end{itemize}
    \end{itemize}
    \begin{itemize}
        \item \emph{Section 2.4 Chain Rule}
        \begin{itemize}
            \item Example 7.
        \end{itemize}
    \end{itemize}

\item \cite{openstax}\footnotemark[2]\textsuperscript{,}\footnotemark[3]
    \begin{itemize}
        \item \emph{Section 3.9 Derivatives of Exponential and Logarithmic Functions}
        \begin{itemize}
            \item Derivative of the Logarithmic Function
        \end{itemize}
    \end{itemize}
\item Review previous lessons:
\begin{itemize}
    \item Lesson \ref{concavity}: Second Derivative and Concavity
    \begin{itemize}
        \item Concavity Test: Determine Concavity and Inflection Values
    \end{itemize}
    \item Lesson \ref{optimization}: Optimization
    \begin{itemize}
        \item Finding Maxima and Minima of a Function
    \end{itemize}
\end{itemize}
\end{itemize}
%\subsection*{Supplemental Materials:}
%%%Key Terms%%%
\subsection*{Key Terms and Concepts:} 

%\begin{multicols}{2}
\begin{itemize}
    \item Differentiation of natural log functions (base $e$).
\end{itemize}
%\end{multicols}
\end{framed}
\footnotetext[1]{Available free to download from \url{http://www.opentextbookstore.com/details.php?id=14} .}
\footnotetext[2]{Available free to download from \url{https://openstax.org/details/books/calculus-volume-1} .}
\footnotetext[3]{Disregard any examples with trigonometry.}

\newpage
%%%%%%%%%%START LESSON CONTENT%%%%%%%%%%%%%
%\noindent\makebox[\linewidth]{\rule{\textwidth}{0.8pt}}
\Opensolutionfile{ans}[ans15]
\Opensolutionfile{ansL}[ansL15]
%%%%%%%%%%%%%%%%Start First Topic%%%%%%%%%%%%%%%%%%%%%%%%%%%%%

\begin{tcolorbox}[title = {The Derivative of the Natural Logarithmic Function}{\hypersetup{linkcolor=white}\footnotemark}]

\noindent If $x>0$ and $y=ln(x)$, then
\begin{equation}\label{eq:dervNlog1}
\frac{dy}{dx}=\frac{1}{x}
\end{equation}
More generally, let $g(x)$ be a differentiable function. For all values of $x$ for which $g'(x)>0$, the derivative of $h(x)=ln(g(x))$ is given by
\begin{equation}\label{eq:dervNlog1}
h'(x)=\frac{1}{g(x)} g'(x)
\end{equation}

\end{tcolorbox}
%%%%%%%%%%%%%Footnotes from Formula Box%%%%%%%%%%%%%%%%%%%%%%
\footnotetext{Theorem 3.15 from \cite{openstax} }
%%%%%%%%%%%%%Footnotes from Formula Box%%%%%%%%%%%%%%%%%%%%%%

%%%Examples%%%
\begin{example}
Given $f(x)=\displaystyle\frac{x}{ln(x)}$, answer the following questions:
\renewcommand{\labelenumi}{\textbf{(\alph{enumi})}}
\begin{enumerate}[leftmargin=*]
\item Find $f'(x)$.\vspace*{\stretch{1}}
\item Find all values of $x$ for which the slope of the tangent line to the graph is equal to 0.\vspace*{\stretch{1}}
\item Find all relative maximum/minimum points. \vspace*{\stretch{1}}
\end{enumerate}
    %%short answer
    \begin{sol}
    \onehalfspacing{
    \begin{enumInline1}
    \item $f'(x)=\displaystyle\frac{ln(x)-1}{[ln(x)]^2}$
    \item $x=e$
    \item Relative minimum point at $(e,e)$
    \end{enumInline1} }
    \end{sol}
    %%solution
    \begin{solL}
    Complete solution here.....
    
    \end{solL}
    
\end{example}
\newpage
%%%%%%%%%%%%%%%%%%%%%%%%
\begin{example}
Given $f(x)=ln\left(\displaystyle\frac{1}{x}\right)$, answer the following questions:
\renewcommand{\labelenumi}{\textbf{(\alph{enumi})}}
\begin{enumerate}[leftmargin=*]
    \item Find $f'(x)$.\vspace*{\stretch{1}}
    \item Find the intervals on which the function is increasing and those on which the function is decreasing.\vspace*{\stretch{1}}
    \item Determine the concavity of the graph and identify all inflection points.\vspace*{\stretch{1}}
\end{enumerate}
    %%short answer
    \begin{sol}
    \onehalfspacing{
    \begin{enumInline1}
    \item $f'(x)=-\displaystyle\frac{1}{x}$
    \item Decreasing on $(0,\infty)$
    \item $f''(x)=\displaystyle\frac{1}{x^2}$; Concave up on $(0,\infty)$; no inflection points.
    \end{enumInline1} }
    \end{sol}
    %%solution
    \begin{solL}
    Complete solution here.....
    
    \end{solL}
    
\end{example}
\newpage
%%%%%%%%%%%%%%%%%%%%%%%%
\begin{example}
Given $f(x)=x\cdot ln(x^2)$, answer the following questions:
\renewcommand{\labelenumi}{\textbf{(\alph{enumi})}}
\begin{enumerate}[leftmargin=*]
    \item Find $f'(x)$.\vspace*{\stretch{1}}
    \item Find the intervals on which the function is increasing and those on which the function is decreasing.\vspace*{\stretch{1}}
    \item Determine the concavity of the graph and identify all inflection points.\vspace*{\stretch{1}}
\end{enumerate}
    %%short answer
    \begin{sol}
    \onehalfspacing{
    \begin{enumInline1}
    \item $f'(x)=2+ln(x^2)$
    \item Increasing on $\left(-\infty,-\displaystyle\frac{1}{e}\right)\cup \left(\displaystyle\frac{1}{e},\infty\right)$
    \item $f''(x)=\displaystyle\frac{2}{x}$; Concave down on $(-\infty,0)$; Concave up on $(0,\infty)$; no inflection points.
    \end{enumInline1} }
    \end{sol}
    %%solution
    \begin{solL}
    Complete solution here.....
    
    \end{solL}
    
\end{example}
\newpage
%%%%%%%%%%%%%%%%%%%%%%%%
\begin{example}
Given $f(x)=[1+ln(x)]^2$, answer the following questions:
\renewcommand{\labelenumi}{\textbf{(\alph{enumi})}}
\begin{enumerate}[leftmargin=*]
    \item Find $f'(x)$.\vspace*{\stretch{1}}
    \item Find the intervals on which the function is increasing and those on which the function is decreasing.\vspace*{\stretch{1}}
    \item Determine the concavity of the graph and identify all inflection points.\vspace*{\stretch{1}}
\end{enumerate}
    %%short answer
    \begin{sol}
    \onehalfspacing{
    \begin{enumInline1}
    \item $f'(x)=\displaystyle\frac{2}{x}(1+ln(x))$
    \item Increasing on  $\left(\displaystyle\frac{1}{e},\infty\right)$; Decreasing on $\left(0,\displaystyle\frac{1}{e}\right)$
    \item $f''(x)=-\displaystyle\frac{2}{x^2}\cdot ln(x)$; Concave up on $(0,1)$; Concave down on $(1,\infty)$; inflection point at $(1,1)$.
    \end{enumInline1} }
    \end{sol}
    %%solution
    \begin{solL}
    Complete solution here.....
    
    \end{solL}
    
\end{example}
\newpage
%%%%%%%%%%%%%%%%%%%%%%%%
\begin{example}
Given $f(x)=ln\left(\displaystyle\frac{x}{x+1}\right)$, answer the following questions:
\renewcommand{\labelenumi}{\textbf{(\alph{enumi})}}
\begin{enumerate}[leftmargin=*]
    \item Find the domain of $f(x)$. \\
    Hint: Reveiw \emph{Finding the Domain of a Logarithmic Function} on \emph{6.4 Graphs of Logarithmic Functions} from \cite{openstaxColAlgebra} \footnotemark[1]. \vspace*{\stretch{1}}
    \item Find $f'(x)$.\vspace*{\stretch{1}}
     \item Find the intervals on which the function is increasing and those on which the function is decreasing.\vspace*{\stretch{1}}
    \item Find all relative maximum/minimum points.\vspace*{\stretch{1}}
\end{enumerate}
    %%short answer
    \begin{sol}
    \onehalfspacing{
    \begin{enumInline1}
    \item $(-\infty,-1)\cup (0,\infty)$
    \item $f'(x)=\displaystyle\frac{1}{x(x+1)}$
    \item Increasing on $(\infty,-1)\cup (0,\infty)$
    \end{enumInline1} }
    \end{sol}
    %%solution
    \begin{solL}
    Complete solution here.....
    
    \end{solL}
    
\end{example}
\footnotetext[1]{Available free to download from \url{https://openstax.org/details/books/college-algebra} .}
%%%%%%%%%%%%End Examples%%%%%%%%%%%%%%%%%%

%%%%%%%%%%%%%%%End Lesson%%%%%%%%%%%%%%%%%%
\Closesolutionfile{ans}
\Closesolutionfile{ansL}

%%%Short Answers to Examples%%%
\newpage
%\vspace*{\fill}

\subsection*{Short Answers to Examples}
%\vspace{-0.25cm}
%\begin{multicols}{2}
\input{ans15}
%\end{multicols}


