%Dave's handout 4.6 Natural Logarithm Functions: Properties and Differentiation 

\vspace{-0.25 in}
\begin{framed}
\subsection*{Objectives}
\begin{itemize}
    \item Be familiar with the properties of log functions and how to use them in solving equations.
    \item Be able to use log properties in differentiating complicated functions that involve log functions.
\end{itemize}

%%%Reading Assignment%%%
\subsection*{Suggested Reading:}
\begin{itemize}
\item \cite{openstaxColAlgebra}\footnotemark[1]
    \begin{itemize}
        \item \emph{Section 6.3 Logarithmic Functions}
        \item \emph{Section 6.5 Logarithmic Properties}
    \end{itemize}
   
\end{itemize}
%\subsection*{Supplemental Materials:}
%%%Key Terms%%%
\subsection*{Key Terms and Concepts:} 

\begin{multicols}{2}
\begin{itemize}
    \item Natural log functions (base $e$)
    \item Properties of log functions.
    \item Differentiation using log properties.
\end{itemize}
\end{multicols}
\end{framed}

\footnotetext[1]{Available free to download from \url{https://openstax.org/details/books/college-algebra} .}

\newpage
%%%%%%%%%%START LESSON CONTENT%%%%%%%%%%%%%
%\noindent\makebox[\linewidth]{\rule{\textwidth}{0.8pt}}
\Opensolutionfile{ans}[ans16]
\Opensolutionfile{ansL}[ansL16]
%%%%%%%%%%%%%%%%Log Properties from Pata's College Algebra handout%%%%%%%%%%%%%%%%%%%%%%%%%%%%%
\begin{comment}

\noindent For any base $a>0$, with $a \neq 1$,
\begin{equation}
\log_a(a) = 1
\end{equation}
\vspace{-0.6cm}
\begin{equation}
\log_a(1) = 0
\end{equation}
\vspace{-0.6cm}
\begin{equation}
\log_a(a^x) =x \quad \text{for any real number} x
\end{equation}
\vspace{-0.6cm}
\begin{equation}
a^{\log_a(x)}=x \quad \text{for any} \quad x>0
\end{equation}
\vspace{-0.6cm}
\begin{equation}
\text{Product Rule: } \log_a(M \cdot N)=log_a(M) +log_a(N)
\end{equation}
\vspace{-0.6cm}
\begin{equation}
\text{Quotient Rule: } \log\left(\frac{M}{N}\right)=\log_a(M) - \log_a(N) 
\end{equation}
\vspace{-0.6cm}
\begin{equation}
\text{Power Rule: } \log_a(M^r) = r \cdot \log_a(M)
\end{equation}
\end{comment}
%%%%%%%%%%%%%%%%%%%%%%%%%%%%%%%%%%%%%%%%%%%%%%%%%%%%%%%%%%%%%%%%%%%%
\noindent The following logarithmic properties are useful in solving equations:
\begin{tcolorbox}[title = {Review: Logarithmic Properties }]

\begin{equation}
\ln(e) = 1
\end{equation}
\vspace{-0.6cm}
\begin{equation}
\ln(1) = 0
\end{equation}
\vspace{-0.6cm}
\begin{equation}
\ln(e^x) =x \quad \text{for any real number} x
\end{equation}
\vspace{-0.6cm}
\begin{equation}
e^{\ln(x)}=x \quad \text{for any} \quad x>0
\end{equation}
\vspace{-0.6cm}
\begin{equation}
\text{Product Rule: } \ln(M \cdot N)=ln(M) +ln(N)
\end{equation}
\vspace{-0.6cm}
\begin{equation}
\text{Quotient Rule: } \ln\left(\frac{M}{N}\right)=\ln(M) - \ln(N) 
\end{equation}
\vspace{-0.6cm}
\begin{equation}
\text{Power Rule: } \ln(M^r) = r \cdot \ln(M)
\end{equation}

\end{tcolorbox}

%%%Examples %%%
\begin{example}
Given $ln(x)+ln(x^2)-ln(x^4)$, simplify the expression and write your answer in terms of $ln(x)$.
    %%short answer
    \begin{sol}
    $-ln(x)$
    \end{sol}
    %%solution
    \begin{solL}
    Complete solution here.....
    
    \end{solL}
    
\end{example}
\vspace*{\stretch{1}}
%%%%%%%%%%%%%%%%%%%%%%%%%%%%%%%%%%%%%%%%%%%%%%%%%%%%%%%%%%%%%
\begin{example}
Which of the following is algebraically equivalent to the expression: $\bm{ln(8x^3)}$? HINT: $8=2^3$\\
\onehalfspacing{
    \begin{enumInline1}
    \item $3\cdot ln(8x)$
    \item $[3\cdot ln(2)]\cdot [3\cdot ln(x)]$
    \item $3\cdot ln(2x)$ 
    \item $3\cdot [ln(8)+ln(x)]$ 
   
    \end{enumInline1} }
    %%short answer
    \begin{sol}
    c
    \end{sol}
    %%solution
    \begin{solL}
    Complete solution here.....
    
    \end{solL}
    
\end{example}
\vspace*{\stretch{1}}
%%%%%%%%%%%%%%%%%%%%%%%%%%%%%%%%%%%%%%%%%%%%%%%%%%%%%%%%%%%%%
\begin{example}
Solve for $x$: $ln(\sqrt{x})-2\cdot ln(3)=0$

    %%short answer
    \begin{sol}
    $x=81$
    \end{sol}
    %%solution
    \begin{solL}
    Complete solution here.....
    
    \end{solL}
    
\end{example}
\vspace*{\stretch{1}}
\newpage
%%%%%%%%%%%%%%%%%%%%%%%%%%%%%%%%%%%%%%%%%%%%%%%%%%%%%%%%%%%%%
\begin{example}
Solve for $x$: $ln(x^2)-ln(2x)+1=0$

    %%short answer
    \begin{sol}
    $x=\displaystyle\frac{2}{e}$
    \end{sol}
    %%solution
    \begin{solL}
    Complete solution here.....
    
    \end{solL}
    
\end{example}
\vspace*{\stretch{1}}
%%%%%%%%%%%%%%%%%%%%%%%%%%%%%%%%%%%%%%%%%%%%%%%%%%%%%%%%%%%%%
\begin{example}
Solve for $x$: $ln(x+4)-ln(x-2)+1=ln(x)$

    %%short answer
    \begin{sol}
    $x=4$
    \end{sol}
    %%solution
    \begin{solL}
    Complete solution here.....
    
    \end{solL}
    
\end{example}
\vspace*{\stretch{1}}
%%%%%%%%%%%%%%%%%%%%%%%%%%%%%%%%%%%%%%%%%%%%%%%%%%%%%%%%%%%%%
\begin{example}
Solve for $x$: $ln(x^3)-4\cdot ln(x)=1$

    %%short answer
    \begin{sol}
    $x=\displaystyle\frac{1}{e}$
    \end{sol}
    %%solution
    \begin{solL}
    Complete solution here.....
    
    \end{solL}
    
\end{example}
\vspace*{\stretch{1}}
%%%%%%%%%%%%%%%%%%%%%%%%%%%%%%%%%%%%%%%%%%%%%%%%%%%%%%%%%%%%%
\begin{example}
Solve for $x$: $ln(x+1)-ln(x)=1$

    %%short answer
    \begin{sol}
    $x=\displaystyle\frac{1}{e-1}$
    \end{sol}
    %%solution
    \begin{solL}
    Complete solution here.....
    
    \end{solL}
    
\end{example}
\vspace*{\stretch{1}}
\newpage
%%%%%%%%%%%%%%%%%%%%%%%%%%%%%%%%%%%%%%%%%%%%%%%%%%%%%%%%%%%%%
\begin{example}
Find the derivative of the function $f(x)=ln\left(\displaystyle\frac{x}{x+1}\right)$.
NOTE:  Use properties of logs to re-write the function before differentiating.	Simply your answer. 

    %%short answer
    \begin{sol}
    $f'(x)=\displaystyle\frac{1}{x(x+1)}$ or \displaystyle\frac{1}{x^2+x}
    \end{sol}
    %%solution
    \begin{solL}
    Complete solution here.....
    
    \end{solL}
    
\end{example}
\vspace*{\stretch{2}}
\begin{example}
Find the derivative of the function $f(x)=ln\left[(x+1)(x^2-1)\right]$.
NOTE:  Use properties of logs to re-write the function before differentiating.	Simply your answer. 

    %%short answer
    \begin{sol}
    $f'(x)=\displaystyle\frac{3x-1}{(x+1)(x-1)}$ or \displaystyle\frac{3x-1}{x^2-1}
    \end{sol}
    %%solution
    \begin{solL}
    Complete solution here.....
    
    \end{solL}
    
\end{example}
\vspace*{\stretch{1}}
%%%%%%%%%%%%End Examples%%%%%%%%%%%%%%%%%%
%%%%%%%%%%%%%%%End Topic%%%%%%%%%%%%%%%%%%



%%%%%%%%%%%%%%%End Lesson%%%%%%%%%%%%%%%%%%
\Closesolutionfile{ans}
\Closesolutionfile{ansL}

%%%Short Answers to Examples%%%
%\newpage
\vspace*{\fill}

\subsection*{Short Answers to Examples}
%\vspace{-0.25cm}
\begin{multicols}{2}
\input{ans16}
\end{multicols}


