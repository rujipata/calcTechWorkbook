%Dave's handout 9.1 Integration by Substitution

\vspace{-0.25 in}
\begin{framed}
\subsection*{Objectives}
\begin{itemize}
    \item Be able to use the method of substitution to determine antiderivatives for complicated functions.
\end{itemize}

%%%Reading Assignment%%%
\subsection*{Suggested Reading:}
\begin{itemize}
\item \cite{Calaway}\footnotemark[1]
   \begin{itemize}
        \item \emph{Section 3.4 Substitution}
    \end{itemize}

\item \cite{openstax}\footnotemark[2]\textsuperscript{,}\footnotemark[3]
    \begin{itemize}
        \item \emph{Section 5.5 Substitution}
    \end{itemize}
\end{itemize}
%\subsection*{Supplemental Materials:}
%%%Key Terms%%%
\subsection*{Key Terms and Concepts:} 

\begin{multicols}{2}
\begin{itemize}
    \item u-substitution 
\end{itemize}
\end{multicols}
\end{framed}
\footnotetext[1]{Available free to download from \url{http://www.opentextbookstore.com/details.php?id=14} .}
\footnotetext[2]{Available free to download from \url{https://openstax.org/details/books/calculus-volume-1} .}
\footnotetext[3]{Disregard any examples with trigonometry.}
\newpage
%%%%%%%%%%START LESSON CONTENT%%%%%%%%%%%%%
%\noindent\makebox[\linewidth]{\rule{\textwidth}{0.8pt}}
\Opensolutionfile{ans}[ans23]
\Opensolutionfile{ansL}[ansL23]
%%%%%%%%%%%%%%%%Start First Topic%%%%%%%%%%%%%%%%%%%%%%%%%%%%%
\noindent While the rules presented in section \ref{IntgSub} are helping in determining antiderivatives of functions, we are limited in the types of functions which we can apply these rules.  For example, consider the following: $\displaystyle\int \left[(x^2-2)^4\cdot 2x\right] dx$. Unless we expand the integrand (unappealing), the rules from section \ref{IntgSub} will not provide a means to evaluate this integral.  We solve this problem by introducing another technique of integration which will allow us, through a substitution to transform the integrand into a form for which the rules from section \ref{IntgSub} will apply.  This technique is often referred to as “\textbf{u-substitution}”.   The method is one way of algebraically manipulating an integrand so that the rules apply. Specifically, this method helps us find antiderivatives when the integrand is the result of a \temph{chain-rule} derivative.\\

\noindent At first, the approach to the substitution procedure may not appear very obvious. However, it is primarily a visual task—--that is, the integrand shows you what to do; it is a matter of recognizing the form of the function. So, what are we supposed to see? We are looking for an integrand of the form $f[g(x)]g'(x)$. \\

\noindent For example, in the integral $\displaystyle\int \left[(x^2-2)^4\cdot 2x\right] dx$, we have $f(x)=x^4$, $g(x)=x^2-2$, and $g'(x)=2x$. Then, $f[g(x)]g'(x)=(x^2-2)^4\cdot 2x$. With substituition, we will substitute $u=g(x)$. This means $\displaystyle\frac{du}{dx}=g'(x)$, so $du=g'(x) dx$. Making this substitutions, $\displaystyle\int \left[(x^2-2)^4\cdot 2x\right] dx=\displaystyle\int u^4 du$, or $\displaystyle\int f[g(x)]g'(x) dx=\int f(u) du$.
\vspace{0.5cm}
\begin{tcolorbox}[title = {Theorem: Substitution with Indefinite Integrals}{\hypersetup{linkcolor=white}\footnotemark}]

\noindent Let $u=g(x)$, where $g'(x)$ is continuous over an interval, let $f(x)$ be continuous over the corresponding range of $g$, and let $F(x)$ be an antiderivative of $f(x)$. Then, 
\begin{equation}
\begin{split}
    \int f[g(x)]g'(x) dx&=\int f(u) du\\
    &=F(u)+C\\
    &=F(g(x))+C.
\end{split}
\end{equation}

\end{tcolorbox}
\vspace{0.5cm}
%%%%%%%%%%%%%Footnotes from Formula Box%%%%%%%%%%%%%%%%%%%%%%
\footnotetext{\cite{openstax}}
%%%%%%%%%%%%%Footnotes from Formula Box%%%%%%%%%%%%%%%%%%%%%%
\begin{tcolorbox}[title={Problem-Solving Strategy: Integration by Substitution}]
\begin{enumerate}[leftmargin=*]
    \item Look carefully at the integrand and select an expression $g(x)$ within the integrand to set equal to $u$. Let's select $g(x)$ such that $g'(x)$ is also part of the integrand.
    \item Substitute $u=g(x)$ and $du=g'(x) dx$ into the integral.
    \item We should now be able to evaluate the integral with respect to $u$. If the integral can't be evaluated we need to go back and select a different expression to use as $u$.
    \item Evaluate the integral in terms of $u$. If you still cannot evaluate it, go back to step 1 and try a different choice of $u$.
    \item Substitute back $x$'s for $u$'s everywhere in your answer.
\end{enumerate}
\end{tcolorbox}
\newpage
%%%%%%%%%%%%%%%%%%Example: Dave's handout 9.1%%%%%%%%%%%%%%%%%%%%%
\begin{example}
Evaluate $\displaystyle\int \left[(x^2-2)^4\cdot 2x\right] dx$.
    %%short answer
    \begin{sol}
    $F(x)=\displaystyle\frac{1}{5}(x^2-2)^5+C$
    \end{sol}
    %%solution
    \begin{solL}
    Complete solution here.....
    
    \end{solL}
    
\end{example}
\vspace*{\stretch{1.25}}
%%%%%%%%%%%%%%%%%%Example: Dave's handout 9.1%%%%%%%%%%%%%%%%%%%%%
\begin{example}
Evaluate $\displaystyle\int \left(3x^2\cdot e^{x^3+2}\right) dx$.
    %%short answer
    \begin{sol}
    $F(x)=e^{x^3+2}+C$
    \end{sol}
    %%solution
    \begin{solL}
    Complete solution here.....
    
    \end{solL}
    
\end{example}
\vspace*{\stretch{1.25}}
%%%%%%%%%%%%%%%%%%Example: Dave's handout 9.1%%%%%%%%%%%%%%%%%%%%%
\begin{example}
Evaluate $\displaystyle\int \left(\frac{e^x}{2+3e^x}\right) dx$.
    %%short answer
    \begin{sol}
    $F(x)=\displaystyle\frac{1}{3}\ln(2+3e^x)+C$
    \end{sol}
    %%solution
    \begin{solL}
    Complete solution here.....
    
    \end{solL}
    
\end{example}
\vspace*{\stretch{1.25}}

%%%%%%%%%%%%%%%%%%Example: Dave's handout 9.1%%%%%%%%%%%%%%%%%%%%%
\begin{example}
Evaluate $\displaystyle\int \left(8x\sqrt{2x^2+4}\right) dx$.
    %%short answer
    \begin{sol}
    $F(x)=\dfrac{(2x^4+7)^8}{8}	+C$
    \end{sol}
    %%solution
    \begin{solL}
    Complete solution here.....
    
    \end{solL}
    
\end{example}
\vspace*{\stretch{1}}
\newpage
%%%%%%%%%%%%%%%%%%Example: Dave's handout 9.1%%%%%%%%%%%%%%%%%%%%%
\begin{example}
Evaluate $\displaystyle\int \left(\frac{\ln(4x)}{x}\right) dx$.
    %%short answer
    \begin{sol}
    $F(x)=\displaystyle\frac{1}{2}[\ln(4x)]^2+C$
    \end{sol}
    %%solution
    \begin{solL}
    Complete solution here.....
    
    \end{solL}
    
\end{example}
\vspace*{\stretch{1.5}}

%%%%%%%%%%%%%%%%%%Example: Dave's handout 9.1%%%%%%%%%%%%%%%%%%%%%
\begin{example}
Evaluate $\displaystyle\int \left(\frac{x}{(x^2+3)^2}\right) dx$.
    %%short answer
    \begin{sol}
    $F(x)=-\displaystyle\frac{1}{2(x^2+3)}+C$
    \end{sol}
    %%solution
    \begin{solL}
    Complete solution here.....
    
    \end{solL}
    
\end{example}
\vspace*{\stretch{1}}
%%%%%%%%%%%%End Examples%%%%%%%%%%%%%%%%%%



%%%%%%%%%%%%%%%End Lesson%%%%%%%%%%%%%%%%%%
\Closesolutionfile{ans}
\Closesolutionfile{ansL}

%%%Short Answers to Examples%%%
%\newpage
\vspace*{\fill}

\subsection*{Short Answers to Examples}
%\vspace{-0.25cm}
%\begin{multicols}{2}
\input{ans23}
%\end{multicols}


