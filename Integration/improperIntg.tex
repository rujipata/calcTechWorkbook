%Dave's handout 9.6 Improper Integrals

\vspace{-0.25 in}
\begin{framed}
\subsection*{Objectives}
\begin{itemize}
    \item Recognize if an integral is an improper integrals.
    \item Evaluate an integral over an infinite interval.
    \item Evaluate an integral over a closed interval with an infinite discontinuity within the interval. 
    \item Apply improper integrals on real-world problems.
\end{itemize}

%%%Reading Assignment%%%
\subsection*{Suggested Reading:}
\begin{itemize}

\item \cite{openstax}\footnotemark[1]\textsuperscript{,}\footnotemark[2]
    \begin{itemize}
        \item \emph{Section 3.7 Improper Integrals}
        \begin{itemize}
            \item Integrating over an Infinite Interval
            \item Integrating a Discontinuous Integrand
        \end{itemize}
    \end{itemize}

\end{itemize}
%\subsection*{Supplemental Materials:}
%%%Key Terms%%%
\subsection*{Key Terms and Concepts:} 

\begin{multicols}{2}
\begin{itemize}
    \item Improper Integrals
    \item Evaluating improper integrals
    \item Convergent improper integrals
    \item Divergent improper integrals
\end{itemize}
\end{multicols}
\end{framed}
\footnotetext[1]{Available free to download from \url{https://openstax.org/details/books/calculus-volume-1} .}
\footnotetext[2]{Disregard \emph{A Comparison Theorem}.}
\newpage
%%%%%%%%%%START LESSON CONTENT%%%%%%%%%%%%%
%\noindent\makebox[\linewidth]{\rule{\textwidth}{0.8pt}}
\Opensolutionfile{ans}[ans26]
\Opensolutionfile{ansL}[ansL26]
%%%%%%%%%%%%%%%%Start First Topic%%%%%%%%%%%%%%%%%%%%%%%%%%%%%
\noindent In this section, we define integrals over an infinite interval as well as integrals of functions containing a discontinuity on the interval. Integrals of these types are called \emph{improper integrals}. There are several techniques for evaluating improper integrals, all of which involve taking limits.

\subsection*{An Imporoper Integral Over An Infinite Interval}


\begin{tcolorbox}[title = {Definition}]
\begin{enumerate}
    \item Let $f(x)$ be continuous over an interval of the form $(a,\infty]$. Then
    \begin{equation}\label{eq:improper1eq}
        \int_a^{\infty} f(x)\ dx = \lim\limits_{t \to \infty} \int_a^t f(x) \ dx
    \end{equation}
   
    \item Let $f(x)$ be continuous over an interval of the form $(-\infty,b]$. Then
    \begin{equation}\label{eq:improper2eq}
        \int_{-\infty}^{b} f(x)\ dx = \lim\limits_{t \to -\infty} \int_t^b f(x) \ dx
    \end{equation}
    
    \item Let $f(x)$ be continuous over $(-\infty,\infty)$. Then
    \begin{equation}\label{eq:improper3eq}
        \int_{-\infty}^{\infty} f(x) \ dx = \int_{-\infty}^a f(x) \dx + \int_a^{\infty} f(x) \dx \ \text{for any value of} \ a
    \end{equation}
\end{enumerate}
\end{tcolorbox}

\noindent In each case in \ref{eq:improper1eq} and in \ref{eq:improper2eq}, if the limit exists, the the improper integral is said to \textbf{converge}. If the limit does not exist then the improper integral is said to \textbf{diverge}.\\

\noindent For case in \ref{eq:improper3eq}, $\int_{-\infty}^{\infty} f(x) \ dx$ \textbf{converges} if both $\int_{-\infty}^a f(x) \dx$ and $\int_a^{\infty} f(x) \dx$ converge. If either one or both of these two integrals diverge, then $\int_{-\infty}^{\infty} f(x) \ dx$ \textbf{diverges}.

\begin{example}
Determine whether the integral below is divergent or convergent. If it is convergent, evaluate it. 

$$\int_{-\infty}^{2} e^x\ dx$$

    %%short answer
    \begin{sol}
    7.3891
    \end{sol}
    %%solution
    \begin{solL}
    Complete solution here.....
    
    \end{solL}
    
\end{example}
%%%%%%%%%%%%%%%%%%%%%%%%%%%%%%%%%%%%%%%%
%%%%%%%%%%%%%%%%%%%%%%%%%%%%%%%%%%%%%%%%
\newpage
\begin{example}
Determine whether the integral below is divergent or convergent. If it is convergent, evaluate it. 

$$\int_{1}^{\infty} \frac{\ln(x)}{x}\ dx$$
\vspace*{\stretch{1}}
    %%short answer
    \begin{sol}
    The integral is divergent.
    \end{sol}
    %%solution
    \begin{solL}
    Complete solution here.....
    
    \end{solL}
    
\end{example}
%%%%%%%%%%%%%%%%%%%%%%%%%%%%
%%%%%%%%%%%%%%%%%%%%%%%%%%%
\begin{example}
Suppose it is known that, on average, 3 customers per minute visit your website. This being the case, you know that the integral

$$\int_{0}^{m} 3\cdot e^{-3t} \ dt$$
will calculate the probability that you will have a customer visit in the next $m$ minutes.\\
\renewcommand{\labelenumi}{(\alph{enumi})}
\begin{enumerate}[leftmargin=*]
    \item What is the percent probability that a customer will visit your website in the next 30 seconds?
 \vspace*{\stretch{1}}
    
    \item Evaluate $\displaystyle\int_0^\infty 3\cdot e^{-3t} \ dt$. Interpret the result.
\vspace*{\stretch{0.5}}
 \end{enumerate}   
 %%short answer
        \begin{sol}
        (a) 77.69\% (b) 1 or 100\% It is a certainty that a customer will eventually visit the website.
        \end{sol}
        %%solution
        \begin{solL}
        Complete solution here.....
    
         \end{solL}
\end{example}

\newpage
\subsection*{An Improper Integral with A Discontinuous Integrand}

\begin{tcolorbox}[title = {Definition}]

\begin{enumerate}
    \item Let $f(x)$ be continuous over an interval of the form $[a,b)$. Then
    \begin{equation}\label{eq:improper4eq}
        \int_a^{b} f(x)\ dx = \lim\limits_{t \to b^-} \int_a^t f(x) \ dx
    \end{equation}
   
    \item Let $f(x)$ be continuous over an interval of the form $(a,b]$. Then
    \begin{equation}\label{eq:improper5eq}
        \int_{a}^{b} f(x)\ dx = \lim\limits_{t \to a^+} \int_t^b f(x) \ dx
    \end{equation}
    
    \item Let $f(x)$ be continuous over $[a,b]$ except at  a point $c$ in $(a,b)$. Then
    \begin{equation}\label{eq:improper6eq}
        \int_{a}^{b} f(x) \ dx = \int_{a}^c f(x) \dx + \int_c^{b} f(x) \dx 
    \end{equation}
\end{enumerate}
\end{tcolorbox}
\noindent In each case in \ref{eq:improper4eq} and in \ref{eq:improper5eq}, if the limit exists, the the improper integral is said to \textbf{converge}. If the limit does not exist then the improper integral is said to \textbf{diverge}.\\
\noindent For case in \ref{eq:improper6eq}, $\int_{a}^{b} f(x) \ dx$ \textbf{converges} if both $\int_{a}^c f(x) \dx$ and $\int_c^{b} f(x) \dx$ converge. If either one or both of these two integrals diverge, then $\int_{a}^{b} f(x) \ dx$ \textbf{diverges}.

%%%%%%%%%%%%%%%%%%%%%%%%
\begin{example}
Determine whether the integral below is divergent or convergent. If it is convergent, evaluate it. 

$$\int_{3}^{8} \frac{1}{\sqrt{8-x}}\ dx$$
\vspace*{\stretch{1}}
    %%short answer
    \begin{sol}
    $2\sqrt{5}$
    \end{sol}
    %%solution
    \begin{solL}
    Complete solution here.....
    
    \end{solL}
    
\end{example}
%%%%%%%%%%%%End Examples%%%%%%%%%%%%%%%%%%




%%%%%%%%%%%%%%%End Lesson%%%%%%%%%%%%%%%%%%
\Closesolutionfile{ans}
\Closesolutionfile{ansL}

%%%Short Answers to Examples%%%
%\newpage
\vspace*{\fill}

\subsection*{Short Answers to Examples}
\input{ans26}


