%Dave's handout 1.7 intro 2nd derivatives
%Dave's handout 2.2 2nd derivative rule
%Second Derivative Test(Determines inflection values and concavity) from https://www.math.tamu.edu/~roquesol/Math142_Spring_2011_Lecture_5.2.pdf; page 3
%Calaway; 2.8; Second Derivative Information; example 4; page 143
%Knewton>Catalog>Business Calculus> Sketch the Curve of a Function (Sketch the graph of a polynomial, Sketch the graph of a rational function)
\vspace{-0.25 in}
\begin{framed}
\subsection*{Objectives}
\begin{itemize}
    \item Understand when and how to use the Product and Quotient rules for finding derivatives
    \item Be able to find derivatives for functions that require the use of more than one rule for differentiation
    \item Second derivatives and concavity
    \item Second derivatives and local max/min.
    \item Second derivatives and inflection points.
    \item Interpret second derivatives in real-world applications (acceleration and diminishing of return)  
\end{itemize}

%%%Reading Assignment%%%
\subsection*{Suggested Reading:}
\begin{itemize}
\item \cite{Calaway}\footnotemark[1]
   \begin{itemize}
        \item \emph{Section 2.3 Power and Sum Rules for Derivatives}
        \begin{itemize}
            \item Continuity 
        \end{itemize}
    \end{itemize}

\item \cite{openstax}\footnotemark[2]\textsuperscript{,}\footnotemark[3]
    \begin{itemize}
        \item \emph{Section 2.4 Continuity}
        \begin{itemize}
            \item Continuity at a Point
            \item Types of Discontinuities
            \item Continuity over an Interval
        \end{itemize}
    \end{itemize}
\item \cite{Hoffman}\footnotemark[3]\textsuperscript{,}\footnotemark[4]
    \begin{itemize}
        \item \emph{Section 1.3: Continuous Functions})
        \begin{itemize}
            \item Definition and Meaning of Continuous.
           \item Graphic Meaning of Continuity
            \item Why do we care whether a function is continuous?
            \item Which Functions Are Continuous?
        \end{itemize}
        
    \end{itemize}
\end{itemize}
%\subsection*{Supplemental Materials:}
%%%Key Terms%%%
\subsection*{Key Terms and Concepts:} 

\begin{multicols}{2}
\begin{itemize}
    \item Rules for finding derivatives of functions
    \item Notation for derivatives of functions
    \item Slope of a graph at a point
\end{itemize}
\end{multicols}
\end{framed}
\footnotetext[1]{Available free to download from \url{http://www.opentextbookstore.com/details.php?id=14} .}
\footnotetext[2]{Available free to download from \url{https://openstax.org/details/books/calculus-volume-1} .}
\footnotetext[3]{Disregard any examples with trigonometry.}
\footnotetext[4]{Available free to download from \url{https://www.opentextbookstore.com/details.php?id=11#tabs-3}.}
\newpage
%%%%%%%%%%START LESSON CONTENT%%%%%%%%%%%%%
%\noindent\makebox[\linewidth]{\rule{\textwidth}{0.8pt}}
\Opensolutionfile{ans}[ans3]
\Opensolutionfile{ansL}[ansL3]
%%%%%%%%%%%%%%%%Start First Topic%%%%%%%%%%%%%%%%%%%%%%%%%%%%%
\noindent Introduction........

\begin{tcolorbox}[title = {Formula Box}]

\noindent In what follows, $f(x)$ and $g(x)$ are differentiable functions of $x$ and $k$ is a constant.\\

\textbf{Constant Multiple Rule:}
\begin{equation}\label{eq:constantMultiple}
\frac{d}{dx}(kf(x))=kf'(x)
\end{equation}

\textbf{Constant Rule:}\footnotemark
\begin{equation}\label{eq:ConstantRule}
\frac{d}{dx}(k)=0
\end{equation}

\end{tcolorbox}
%%%%%%%%%%%%%Footnotes from Formula Box%%%%%%%%%%%%%%%%%%%%%%
\footnotetext{The derivative of a constant is zero because \(k=kx^0\).}
%%%%%%%%%%%%%Footnotes from Formula Box%%%%%%%%%%%%%%%%%%%%%%
\subsection*{Topic 1}
content here.....
%%%Examples%%%
\begin{example}
example content.......
    %%short answer
    \begin{sol}
    short answer here.....
    \end{sol}
    %%solution
    \begin{solL}
    Complete solution here.....
    
    \end{solL}
    
\end{example}
\vspace{0.6in}
%%%%%%%%%%%%%%%%%%%%%%%%
\begin{example}
example content.......
    %%short answer
    \begin{sol}
    short answer here.....
    \end{sol}
    %%solution
    \begin{solL}
    Complete solution here.....
    
    \end{solL}
    
\end{example}
\vspace{0.6in}
%%%%%%%%%%%%End Examples%%%%%%%%%%%%%%%%%%
%%%%%%%%%%%%%%%End Topic%%%%%%%%%%%%%%%%%%
%%%%%%%%%%%%%%%%Begin Next Topic%%%%%%%%%%%%%%%%%%%%%%%%%%%%%%%
\vspace{2in}
\subsection*{Topic 2}
content here.....
%%%Examples%%%
\begin{example}
example content.......
    %%short answer
    \begin{sol}
    short answer here.....
    \end{sol}
    %%solution
    \begin{solL}
    Complete solution here.....
    
    \end{solL}
    
\end{example}
\vspace{0.6in}
%%%%%%%%%%%%%%%%%%%%%%%%
\begin{example}
example content.......
    %%short answer
    \begin{sol}
    short answer here.....
    \end{sol}
    %%solution
    \begin{solL}
    Complete solution here.....
    
    \end{solL}
    
\end{example}
\vspace{0.6in}
%%%%%%%%%%%%End Examples%%%%%%%%%%%%%%%%%%
%%%%%%%%%%%%%%%End Topic%%%%%%%%%%%%%%%%%%



%%%%%%%%%%%%%%%End Lesson%%%%%%%%%%%%%%%%%%
\Closesolutionfile{ans}
\Closesolutionfile{ansL}




